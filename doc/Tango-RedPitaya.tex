\documentclass[12pt,a4paper]{article}
\usepackage[utf8]{inputenc}
\usepackage{polski}
\usepackage{graphicx}

\title{Tango-RedPitaya\\\small{Tango device server for RedPitaya multi-instrument board}}
\author{Grzegorz Kowalski\\\texttt{daneos@daneos.com}}
\date{\today}

\begin{document}
	\maketitle
	\vspace{4em}
	
	\begin{abstract}
		\textbf{Tango-RedPitaya} jest serwerem urządzeń Tango sterującym płytką RedPitaya.\\
		Serwer napisany jest w języku Python i był testowany na wersji 2.7.6 pod kontrolą systemu GNU\textbackslash Linux.\\
		Celem tego projektu jest dostarczenie pełnej funkcjonalności urządzenia RedPitaya przez interfejs Tango.
	\end{abstract}

	\newpage
	\section{Przygotowanie środowiska}
	Po stronie komputera program wymaga zainstalowanego interpretera języka Python oraz pakietów PyTango oraz PyRedPitaya, które są dostępne przez PyPI.\\
	Po stronie płytki RedPitaya musi być zainstalowany Python wraz z pakietem PyRedPitaya. Instrukcja instalacji znajduje się na stronie\\
	\texttt{https://github.com/clade/RedPitaya/blob/master/python/README.rst}.\\
	Dostępna jest również wersja obrazu karty SD z preinstalownym środowiskiem PyRedPitaya:\\
	\texttt{http://clade.pierre.free.fr/python-on-red-pitaya/ecosystem-0.92-0-devbuild.zip}\\
	Po zainstalowaniu odpowiedniego środowiska na karcie SD, należy wprowadzić modyfikacje dostępne wraz z programem w katalogu \texttt{src/rp\_board}.
	Pliki z tego katalogu należy skopiować w odpowiednie miejsca karty SD:
	\begin{itemize}
		\item \texttt{rcS} $\to$ \texttt{etc/init.d/} -- uruchamia serwer RPyC podczas startu systemu
		\item \texttt{service.py} $\to$ \texttt{usr/lib/python2.7/site-packages/PyRedPitaya/} -- udostępnia usługę zdalnego uruchamiania komend
	\end{itemize}
	Dostępna jest wersja obrazu karty SD z wprowadzonymi wszystkimi zmianami:\\
	\emph{tu adres strony}\\
	W przypadku korzystania z tego obrazu, wystarczy go zainstalować na karcie SD i uruchomić urządzenie.

	\section{Konfiguracja}
	Urządzenie wykorzystuje trzy własności służące do konfiguracji: \texttt{host}, \texttt{port} i \texttt{reconnect}.\\
	\texttt{Host} jest adresem płytki RedPitaya w sieci.\\
	\emph{tu screenshot}\\
	Domyślnym portem, na którym działa usługa jest \texttt{18861}. Można to zmienić w pliku \texttt{src/rp\_board/service.py}, pamiętając o skopiowaniu pliku na kartę SD po dokonanych modyfikacjach.\\
	\emph{tu screenshot}\\
	Parametr \texttt{reconnect} określa maksymalną liczbę prób połączenia w przypadku jego zerwania. Domyślną wartością jest \texttt{10}.\\
	\emph{tu screenshot}

	\section{Atrybuty}
	Atrybuty dostępne w widoku OPERATOR:
	\begin{itemize}
		\item \textbf{FPGA Temperature} -- aktualna temperatura chipu FPGA\\
			  \texttt{temperature}\\
			  Dostęp: odczyt\\
			  Jednostka: $^{\circ}$C
		\item \textbf{LED state} -- stan wskaźników LED\\
			  \texttt{leds}\\
			  Dostęp: odczyt/zapis\\
			  Zakres wartości: 0-255
		\item \textbf{Ping check} -- stan połączenia z płytką\\
			  \texttt{ping}\\
			  Dostęp: odczyt
		\item \textbf{Scope active} -- stan oscylockopu\\
			  \texttt{scope\_active}\\
			  Dostęp: odczyt
		\item \textbf{Generator CH1 active} -- stan kanału 1 generatora\\
			  \texttt{generator\_ch1\_active}\\
			  Dostęp: odczyt
		\item \textbf{Generator CH2 active} -- stan kanału 2 generatora\\
			  \texttt{generator\_ch2\_active}\\
			  Dostęp: odczyt
	\end{itemize}
	\emph{tu screenshot}\\
	Atybuty dostępne w widoku EXPERT:
	\begin{itemize}
		\item \textbf{PINT Voltage} -- napięcie wewnętrzne jednostki przetwarzania\\
			  \texttt{pint\_voltage}\\
			  Dostęp: odczyt\\
			  Jednostka: V\\
		\item \textbf{PAUX Voltage} -- napięcie pomocnicze jednostki przetwarzania\\
			  \texttt{paux\_voltage}\\
			  Dostęp: odczyt\\
			  Jednostka: V\\
		\item \textbf{BRAM Voltage} -- napięcie bloków pamięci RAM\\
			  \texttt{bram\_voltage}\\
			  Dostęp: odczyt\\
			  Jednostka: V\\
		\item \textbf{INT Voltage} -- napięcie wewnętrzne jednostki logiki programowalnej\\
			  \texttt{int\_voltage}\\
			  Dostęp: odczyt\\
			  Jednostka: V\\
		\item \textbf{AUX Voltage} -- napięcie pomocnicze jednoski logiki programowalnej\\
			  \texttt{aux\_voltage}\\
			  Dostęp: odczyt\\
			  Jednostka: V\\
		\item \textbf{DDR Voltage} -- napięcie DDR I/O\\
			  \texttt{ddr\_voltage}\\
			  Dostęp: odczyt\\
			  Jednostka: V\\
	\end{itemize}
	\emph{tu screenshot}

	\section{Komendy}

	\section{Stan}

	\section{Status}

	\section{Budowa wewnętrzna serwera}

	\section{Uwagi}

\end{document}